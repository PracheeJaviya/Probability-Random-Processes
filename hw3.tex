\documentclass{article}
%\usepackage[a4paper, total={6in, 8in}]{geometry}
\usepackage{geometry}
 \geometry{
 a4paper,
 total={210mm,297mm},
 left=20mm,
 right=20mm,
 top=-2mm,
 bottom=2mm,
 }
%\usepackage[margin=0.5in]{geometry}
\usepackage{mathtools}
\usepackage{amsmath,amssymb}
\usepackage{ifpdf}
%\usepackage{cite}
\usepackage{algorithmic}
\usepackage{array}
\usepackage{mdwmath}
\usepackage{pdfpages}
\usepackage{mdwtab}
\usepackage{eqparbox}
%\onecolumn
%\input{psfig}
\usepackage{color}
\usepackage{graphicx}
\setlength{\textheight}{23.5cm} \setlength{\topmargin}{-1.05cm}
\setlength{\textwidth}{6.5in} \setlength{\oddsidemargin}{-0.5cm}
\renewcommand{\baselinestretch}{1}
\pagenumbering{arabic}

\begin{document}
\textbf{
  \begin{center}
    {
      \large{School of Engineering and Applied Science (SEAS), Ahmedabad University}\vspace{5mm}
    }
  \end{center}
  %
  \begin{center}
    \large{BTech(ICT) Semester IV :Probability and Random Processes(MAT202)\\ \vspace{4mm}
      Homework Assignment-3\\\vspace{2mm}
      Enrollment No:AU1841032 \hspace{6cm} Name:Prachee Javiya  }
\end{center}}
\vspace{5mm}

\begin{enumerate}
\item
  Given :
  $$f_{X|M=0}(x)=\frac{1}{\sqrt{2\pi\sigma^2}}exp(-\frac{x^2}{2\sigma^2})$$ and
  $$f_{X|M=1}(x)=\frac{1}{\sqrt{2\pi\sigma^2}}exp(-\frac{(x-m)^2}{2\sigma^2})$$
  \\
  Since messages sent are equiprobable , we have Pr(M=0)=$\frac{1}{2}$ and Pr(M=1)=$\frac{1}{2}$
  \\
  \\
  a) To find Pr( M=0 $|$ X= x) :
  \hspace{2mm}
  \\
  \\ Given $\sigma^2=1$

  \begin{equation}
    \begin{split}
      Pr( M=0| X= x) & = \frac{Pr( X = x | M=0 ) \times Pr(M=0)}{Pr(X = x)}\\ \\
	  & = \frac{Pr(X = x| M= 0 )\times Pr(M = 0)}{Pr(X=x| M = 0 ) \times Pr(M=0) + Pr(X=x| M = 1 ) \times Pr(M = 1)}\\ \\
	  & =\frac{ \frac{1}{\sqrt{2\pi}}e^{-\frac{x^2}{2}} \times \frac{1}{2}}{ \frac{1}{\sqrt{2\pi}}e^{-\frac{x^2}{2}} \times \frac{1}{2} +  \frac{1}{\sqrt{2\pi}}e^{-\frac{(x-1)^2}{2}} \times \frac{1}{2}} \\ \\
	  & = \frac{e^{\frac{-x^{2}}{2}}}{e^{-\frac{x^2}{2}}+e^{-\frac{(x-1)^2}{2}}}\\ \\
	  & = \frac{1}{1+e^\frac{{2x-1}}{2}}
    \end{split}
  \end{equation}
  \newpage
  For $\sigma^{2}=5 $

  \begin{equation}
    \begin{split}
      Pr( M=1| X= x) & = \frac{Pr( X = x | M=0 ) \times Pr(M=0)}{Pr(X = x)}\\ \\
	  & = \frac{Pr(X = x| M= 0 )\times Pr(M = 0)}{Pr(X=x| M = 0 ) \times Pr(M=0) + Pr(X=x| M = 1 ) \times Pr(M = 1)}\\ \\
	  & =\frac{ \frac{1}{\sqrt{10\pi}}exp(-\frac{x^2}{10}) \times \frac{1}{2}}{ \frac{1}{\sqrt{10\pi}}exp(-\frac{x^2}{10}) \times 				\frac{1}{2} +  \frac{1}{\sqrt{10\pi}}exp(-\frac{(x-1)^2}{10}) \times \frac{1}{2}} \\ \\
	  & = \frac{e^{\frac{-x^{2}}{10}}}{e^{-\frac{x^2}{10}}+e^{-\frac{(x-1)^2}{10}}}\\ \\
	  & = \frac{1}{1+e^\frac{{2x-1}}{10}}
    \end{split}
  \end{equation}



  b) Now given,
  Pr(M=0)=$\frac{1}{4}$ and Pr(M=1)=$\frac{3}{4}$\\
  \\
  \begin{equation}
    \begin{split}
      Pr( M=0| X= x) & = \frac{Pr( X = x | M=0 ) \times Pr(M=0)}{Pr(X = x)}\\ \\
	  & = \frac{Pr(X = x| M= 0 )\times Pr(M = 0)}{Pr(X=x| M = 0 ) \times Pr(M=0) + Pr(X=x| M = 1 ) \times Pr(M = 1)}\\ \\
	  & =\frac{ \frac{1}{\sqrt{2\pi}}e^{-\frac{x^2}{2}} \times \frac{1}{4}}{ \frac{1}{\sqrt{2\pi}}e^{-\frac{x^2}{2}} \times \frac{1}{4} +  \frac{1}{\sqrt{2\pi}}e^{-\frac{(x-1)^2}{2}} \times \frac{3}{4}} \\ \\
	  & = \frac{e^{\frac{-x^{2}}{2}}\times \frac{1}{4}}{e^{-\frac{x^2}{2}} \times \frac{1}{4}+e^{-\frac{(x-1)^2}{2}} \times \frac{3}{4}}\\ \\
	  & = \frac{1}{1+3e^\frac{{2x-1}}{2}}
    \end{split}
  \end{equation}
  \newpage
  For $\sigma^2$=5,
  \\
  \begin{equation}
    \begin{split}
      Pr( M=0| X= x) & = \frac{Pr( X = x | M=0 ) \times Pr(M=0)}{Pr(X = x)}\\ \\
	  & = \frac{Pr(X = x| M= 0 )\times Pr(M = 0)}{Pr(X=x| M = 0 ) \times Pr(M=0) + Pr(X=x| M = 1 ) \times Pr(M = 1)}\\ \\
	  & =\frac{ \frac{1}{\sqrt{10\pi}}e^{-\frac{x^2}{10}} \times \frac{1}{4}}{ \frac{1}{\sqrt{10\pi}}e^{-\frac{x^2}{10}} \times 				\frac{1}{4} +  \frac{1}{\sqrt{10\pi}}e^{-\frac{(x-1)^2}{10}} \times \frac{3}{4}} \\ \\
	  & = \frac{e^{\frac{-x^{2}}{10}}\times \frac{1}{4}}{e^{-\frac{x^2}{10}} \times \frac{1}{4}+e^{-\frac{(x-1)^2}{10}} \times \frac{3}{4}}\\ \\
	  & = \frac{1}{1+3e^\frac{{2x-1}}{10}}
    \end{split}
  \end{equation}

\item
  Assuming the two messages are equiprobable : Pr(M=1)=Pr(M=0)= $\frac{1}{2}$\\
  From the above question, we found Pr(M=0 $|$ X= x)=$\frac{1}{1+e^\frac{2x-1}{2}}$\\
  It is a decreasing function.\\

  To find x , we equate it to 0.9\\
  $$ \frac{1}{1+e^\frac{{2x-1}}{2}} \geq \frac{9}{10}$$
  $$1+e^\frac{{2x-1}}{2} \geq \frac{10}{9}$$
  $$e^\frac{2x-1}{2} \geq \frac{1}{9}$$
  $$ x \leq \frac{2ln(\frac{1}{9})+1}{2}$$
  $$    \leq -1.67 $$
  $\therefore$  Pr(M=0  $ | $ X= x) $\geq$ 0.9 for x ranging between $-\infty$ to -1.697 \\ \\

  $$ \frac{1}{1+e^\frac{{-2x+1}}{2}} \geq \frac{9}{10}$$
  $$1+e^\frac{{-2x+1}}{2} \geq \frac{10}{9}$$
  $$e^\frac{-2x+1}{2} \geq \frac{1}{9}$$
  $$ x \leq -\frac{2ln(\frac{1}{9})-1}{2}$$
  $$ \leq -1.67 $$
  $\therefore$  Pr(M=1  $ | $ X= x) $\geq$ 0.9 for x ranging between $-\infty$ to 2.698\\ \\

  Decide not to decide Pr(M = 0 $|$ X = x) > 0.9 and Pr(M = 1 $|$ X = x)$ >$ 0.9
  Range will be the intersection of previous both ranges i.e. (-1.6972,2.6972)


  b) The probability that the receiver erases a symbol is given by :
  \begin{align*}
    Pr(erased) =& Pr(erased|M = 0)Pr(M = 0)+Pr(erased|M = 1)Pr(M = 1)\\
    =& \dfrac{Pr(erased|M = 0) + Pr(erased|M = 1)}{2}\\
    =& \dfrac{Pr(-1.6972 < x< 2.6972|M = 0) + Pr(-1.6972 < x< 2.6972|M = 1)}{2}\\
    =& \dfrac{[Q(-1.6972)-Q(2.6972)] + [Q(-1.6972-1)-Q(2.6972-1)]}{2}\\
    =& \dfrac{2-2Q(1.6972)-2Q(2.6972)}{2}\\
    =& 1-Q(1.6972)-Q(2.6972)\\
    \approx& 0.95197
  \end{align*}
  \textit{* using properties of Q-function = Q$\left(\dfrac{x-\mu}{\sigma}\right)$ and Q-table}\\

  c)  The probability that the receiver makes an error is given by;
  \begin{align*}
    Pr(error) =& Pr(error|M = 0)Pr(M = 0)+Pr(error|M = 1)Pr(M = 1)\\
    =& \dfrac{Pr(error|M = 0) + Pr(error|M = 1)}{2}\\
    =& \dfrac{Pr(x \geq 2.6972|M = 0) + Pr(x \leq -1.6972|M = 1)}{2}\\
    =& \dfrac{[Q(2.6972)] + [1-Q(-1.6972-1)]}{2}\\
    =& \dfrac{2Q(2.6972)}{2}\\
    =& Q(2.6972)\\
    \approx& 0.003467
  \end{align*}

\item
  PDF of a Rayleigh RV is given as :
  \large $$f(r,\sigma) = \frac{r}{\sigma^2} e^{-\frac{r^2}{2\sigma^2}}$$

  R1=3/2=1.5 ft\\
  R2=0.5/2=0.25 ft \\
  $\sigma^2=4$\\
  \large
  a)Probability of Mr Hood hitting the target: Pr(R $\leq$ R1) = $\int_{0}^{R1}f_{R}(r)dr$\\
  $$=\int_{0}^{R1} \frac{r}{\sigma^2} e^{-\frac{r^2}{2\sigma^2}}$$
  Replacing $r^2$ by x and 2rdr by dx, we get :\\
  $$\int_{0}^{2.25}\frac{1}{2\sigma^2} e^{-\frac{x}{2\sigma^2}}dx$$
  $$Pr(R \leq R1) = 1-e^{-\frac{R1}{2\sigma^2}}$$
  $$= 1-e^{-\frac{R1^2}{8}}$$
  Limits will change from 1.5 to 2.25 while substitution\\
  $$= 0.24 $$

  b)Probability of hitting bulls eye :\\
  Pr(R $\leq$ R2) = $\int_{0}^{R2}f_{R}(r)dr$\\
  \large
  $$=\int_{0}^{R2} \frac{r}{\sigma^2} e^{-\frac{r^2}{2\sigma^2}}$$
  Replacing $r^2$ by x, we get :\\
  $$\int_{0}^{0.0625}\frac{1}{2\sigma^2} e^{-\frac{x}{2\sigma^2}}dx$$
  $$Pr(R \leq R2) = 1-e^{-\frac{R2^2}{2\sigma^2}}$$
  $$= 1-e^{-\frac{R2^2}{8}}$$
  Limits will change from 0.25 to 0.0625\\
  $$= 0.007 $$

  c)Probability that Mr Hood hit bulls eye given he hit the target :\\ \\
  \large $$= \frac{\text{Probability of hitting bulls eye} \cap \text{Probability of hitting target}}{\text{Probability of hitting target}}$$
  Pr(Hitting bulls-eye) $\subset $ Pr(hitting target)
  \large$$ = \frac{\text{Probability of hitting bulls eye}}{\text{Probability of hitting target}} $$
  \\
  \large$$ = \frac{0.007}{0.245} = 0.0317$$
  \newpage
\item
  PDF of Gaussian random variable :

  $$f_{X}(x)= \frac{1}{\sqrt{2\pi\sigma^2}}e^{(-\frac{(x-m)^2}{2\sigma^2})}$$

  Odd central moments are given y $E[(X-\mu_{X})^k]$ where k= 2x-1 and x $\in$ N and $\mu_{x}=E[X]$

  $$\therefore E[(X-\mu_{X})^k] = \frac{1}{\sqrt{2\pi\sigma^2}} \int_{-\infty}^{\infty} (x-\mu)^k  e^{(-\frac{(x-m)^2}{2\sigma^2})} dx$$
  Substituting , $\frac{x-\mu}{\sigma}=t$, we get ,
  $$E[(X-\mu_{X})^k] = \frac{\sigma^k}{\sqrt{2\pi}}\int_{-\infty}^{\infty} t^{k} e^{\frac{-t^2}{2}} dt$$

  When k is odd, the function becomes odd over $-\infty$ to $\infty$ and total area becomes zero.\\
  When k is even,

  $$I_{k} = \int_{-\infty}^{\infty} t^{k} e^{-\frac{t^2}{2}} dt $$
  Integrating by parts, x = $t^{k-1}$ and dy = $te^{-\frac{t^2}{2}}dt$
  $$I_{k}=(k-1)\int_{-\infty}^{\infty} t^{k-2} e^{-\frac{t^2}{2}} $$
  $$= (k-1)I_{k-2}$$
  Following the above steps, it is observed that ,
  $$I_{k}= (k-1)(k-3)(k-5)...3.1.I_{0}$$
  $$I_{0}=\sqrt{2\pi}$$
  For k $\geq$ 2,
  $$E[(X-\mu_{X})^k] = \sigma^{k} \frac{I_{k}}{I_{0}}$$
  $$E[(X-\mu_{X})^k] = \frac{\sigma^2 k!}{(k/2)!2^{k/2}}$$

  \newpage
\item
  Given PDF  a gaussian random variable :
  $$f_{X}(x)= c e^{-(2x^2+3x+1)}$$
  To bring it into gaussian form, we will convert the quadratic equation into whole square form:\\
  \\
  $2x^2+3x+1$ \\ \\$= 2(x^2+\frac{3}{2}x+\frac{1}{2})$\\ \\
  $=2(x^2+2.\frac{3}{4}x+\frac{1}{2})$\\ \\
  $=2(x^2+2.\frac{3}{4}x+\frac{9}{16}-\frac{9}{16}+\frac{1}{2})$ \\ \\
  $= 2((x+\frac{3}{4})^2 - (\frac{1}{4})^2$ \\ \\
  Our PDF, after multiplying and dividing 2 in the numerator and denominator,will now look like, :\\
  \Large $$c e^{\frac{-4((x+\frac{3}{4})^2 - (\frac{1}{4})^2)}{2}}$$
  $$=c . e^{\frac{1}{8}} e ^{-\frac{(x+\frac{3}{4})^2}{2.\frac{1}{4}}}$$\\
  \\
  \large
  From the equation, $\mu= -\frac{3}{4}$ and $\sigma = \frac{1}{2}$\\
  For the coefficient :
  $$\int_{-\infty}^{\infty} f_{X}(x) =1$$
  $$=c.e^{\frac{1}{8}} \int_{-\infty}^{\infty} e^{-2(x+\frac{3}{4})^2}dx = 1$$
  $$=c.e^{\frac{1}{8}}\sqrt{\frac{\pi}{2}}=1 $$
  $$\therefore c= \sqrt{\frac{2}{\pi}}e^{-\frac{1}{8}}$$
\item

  For a Gaussian Random Variable,
  $$f_{X}(x) = \frac{1}{\sqrt{2\pi\sigma^2}}exp(-\frac{(x-\mu)^2}{2\sigma^2})$$
  \newline
  $c_s = E[(\frac{x - \mu}{\sigma})^3]$ and $c_k = E[(\frac{x - \mu}{\sigma})^4]$
  \newline
  \newline
  let, $\frac{x - \mu}{\sigma} = y$
  $$E[(Y)^3] = \frac{1}{\sqrt{2\pi}}\int_{-\infty}^{\infty} y^{3} exp(-\frac{y^2}{2\sigma^2})dy$$
  \newline
  Evaluates to Zero as the given integral is odd function. Hence,
  $$c_s = 0$$
  $$E[(Y)^4] = \frac{1}{\sqrt{2\pi}}\int_{-\infty}^{\infty} y^{4} exp(-\frac{y^2}{2\sigma^2})dy$$
  $$E[(Y)^4] = \frac{1}{\sqrt{2\pi}}\int_{-\infty}^{\infty} y^{3}\hspace{1mm}y\hspace{1mm} exp(-\frac{y^2}{2\sigma^2})dy$$
  Evaluating the Integral using by parts we get
  $$E[(Y)^4] = \frac{3}{\sqrt{2\pi}}\int_{-\infty}^{\infty} y^{2} exp(-\frac{y^2}{2\sigma^2})dy$$
  $$= 3\sigma^2$$

  \newpage
\item Given \\
  $$U = Xcos\theta - Ysin\theta $$
  $$V = Xsin\theta +Ycos\theta$$
  Here, X and Y are independaent, $\mu =0 $, $\sigma=1$ , Gaussian Random Variable
  Joint PDF of U and V is :

  $$ f_{U,V}(u,v)=\frac{f_{X,Y}}{\left|J\left(\begin{matrix}
      u & v \\
      x & y
    \end{matrix}\right)\right|}
  $$
  In the above equation ,
  $$
  \left|J\left(\begin{matrix}
    u & v \\
    x & y
  \end{matrix}\right)\right| \text{is the Jacobian matrix} $$
  \\
  Now we need to find a function for X and Y in terms of U and V \\
  \\
  Multiplying equation 1 with $cos\theta$ on both sides,
  $$Ucos\theta = Xcos^2\theta-Ysin\theta cos\theta ... (3)$$
  Multiplying $sin\theta$ on both sides of equation 2,
  $$Vsin\theta=Xsin^2\theta + Ysin\theta cos\theta ... (4)$$
  Adding eq 3 and 4,
  $$X=U cos\theta +Vsin\theta$$
  Subtracting eq 3 and 4,
  $$Y=Vcos\theta -U sin\theta$$
  \\
  $$
  \left|J\left(\begin{matrix}
    u & v \\
    x & y
  \end{matrix}\right)\right| =
  \left|
  \begin{matrix}
    \frac{\partial z}{\partial x} &  \frac{\partial z}{\partial y}\\
    \frac{\partial v}{\partial x} &  \frac{\partial v}{\partial y}
  \end{matrix}\right|
  =\left|
  \begin{matrix}
    cos\theta & -sin\theta \\ sin\theta & cos\theta
  \end{matrix}
  \right|
  $$
  $$= sin^2\theta+cos^2\theta = 1$$
  \\
  Hence, the joint PDF of U and V  is equal to joint PDF of X and Y.\\

  \newpage

\item
  Given pdf of the random variable :
  $$f_{X}(x)=ce^{-2x}u(x)$$
  Applying the property of PDF :
  $$\int_{-\infty}^{\infty}f_{X}(x)dx=1$$
  $$c\int_{0}^{\infty}e^{-2x}dx=1$$
  $$-\frac{c}{2}[e^{-2x}]_{0}^{\infty} = 1$$
  $$ -\frac{c}{2}[-1]=1$$
  $$\therefore c=2$$
  \\
  b) $$Pr(X>2)= \int_{2}^{\infty}f_{X}(x)dx$$
  $$ \int_{2}^{\infty}2e^{-2x}dx $$
  $$ = e^{-4}$$
  \\
  c) $$Pr(X<3) = \int_{0}^{3}2e^{-2x}dx $$
  $$ = -[e^{-2x}]_{0}^{3}$$
  $$= 1-e^{-6}$$\\
  d) $$ Pr(X<3 | X>2) = \frac{ Pr(2<X<3)}{Pr(X>2)}$$
  $$Pr(2<X<3) = e^{-4}-e^{-6}$$
  $$ \frac{e^{-4}-e^{-6}}{e^{-4}}$$
  $$= 1-e^{-2}$$

  \newpage
\item
  We have a Random variable X which has a uniform distribution over the interval (-a,a) for some positive constant a. Its PDF will be,
  \begin{align*}
    f_{X}(x)&=\frac{1}{b-a}\\
    &=\frac{1}{a-(-a)}\\
    &=\frac{1}{2a}
  \end{align*}
  Now,
  \begin{align*}
    E[X]&=\int_{-a}^{a} x \left( \frac{1}{2a} \right) dx\\
    &=\frac{1}{2a}[0]\\
    &=0
  \end{align*}
  To find the variance we will calculate $E[X^2]$,
  \begin{align*}
    E[X^2]&=\int_{-a}^{a} x^2 \left( \frac{1}{2a} \right) dx\\
    &=\frac{1}{6a}x^3\Big|_{-a}^{a} \\
    \sigma^2 &=\frac{1}{3}a^2
  \end{align*}



  For coefficient of skewness
  \begin{align*}
    E[X^3] &=\int_{-a}^{a} x^3 \left( \frac{1}{2a} \right) dx \\
    &=0
  \end{align*}
  \begin{align*}
    c_{s}&=\frac{E[X^3]}{\sigma^3}\\
    &=0
  \end{align*}
  Now for coefficient of kurtosis
  \begin{align*}
    E[X^4]&=\int_{-a}^{a}\frac{x^4}{2a}dx\\
    &=\frac{1}{10a}x^5\Big|_{-a}^{a} \\
    &=\frac{1}{5}a^4
  \end{align*}
  \begin{align*}
    c_{s}&=\frac{E[X^4]}{\sigma^4}\\
    &=\frac{\frac{1}{5}a^4}{ \left( \frac{1}{3}a^2 \right)^2 }\\
    &=\frac{9}{5}
  \end{align*}

  \vspace{2mm}
  \newpage
\item

  $X$ is a continuous random variable uniform on $[1, 10]$. Hence we know the PDF and CDF of X.\\
  The PDF of X:\\
  $f_X(x) =  {  \left\{
    \begin{array}{ll}
      {\dfrac{1}{b-a} = \dfrac{1}{9}}&  1\leq x\leq 10 \\\\
      0 & otherwise \\
    \end{array}
    \right. }$\\\\\\
  The CDF of X:\\
  $F_X(x) =  {  \left\{
    \begin{array}{ll}
      0 & x\leq 1 \\\\
      {\dfrac{x-a}{b-a} = \dfrac{x-1}{9} }&  1\leq x\leq 10 \\\\
      1 & x\geq 10 \\
    \end{array}
    \right. }$\\\\
  To find the PDF of $\sqrt{X}$, we will find its CDF and differentiate it.\\
  Let, $Y = \sqrt{X}$
  \begin{align*}
    F_Y(y)\hspace{1mm} =& \hspace{2mm}P(Y \leq y)\\
    =& \hspace{2mm}P(\sqrt{X} \leq y)\\
    =& \hspace{2mm}P(X \leq y^2)\\
    =& \hspace{2mm}F_X(y^2)
  \end{align*}
  Differentiating both side,
  \begin{align*}
    f_Y(y) &= 2.y.f_X(y^2)\\
    &= \dfrac{2y}{9}
  \end{align*}

  The PDF of Y is\\
  $f_Y(y) =  {  \left\{
    \begin{array}{ll}
      {\dfrac{2y}{9}}&  1\leq x\leq \sqrt{10} \\\\
      0 & otherwise \\
    \end{array}\right. }$\\\\
  To find the PDF of $ \text{-} \ln{X}$, we will find its CDF and differentiate it.\\
  Let, $Y = \text{-}\ln{X}$
  \begin{align*}
    F_Y(y)\hspace{1mm} =& \hspace{2mm}P(Y \leq y)\\
    =& \hspace{2mm}P(\text{-}\ln{X} \leq y)\\
    =& \hspace{2mm}P(X \geq e^{\text{-}y})\\
    =& \hspace{2mm}1 - P(X \leq e^{\text{-}y})\\
    =& \hspace{2mm}1 - F_X(e^{\text{-}y})
  \end{align*}
  Differentiating both side,
  \begin{align*}
    f_Y(y) &= e^{\text{-}y}f_X(e^{\text{-}y})\\
    &= \dfrac{e^{\text{-}y}}{9}
  \end{align*}
  The PDF of Y is\\
  $f_Y(y) =  {  \left\{
    \begin{array}{ll}
      {\dfrac{e^{\text{-}y}}{9}}&  -\ln{10}\leq x\leq 0 \\\\
      0 & otherwise \\
    \end{array}\right. }$

\end{enumerate}
\end{document}
